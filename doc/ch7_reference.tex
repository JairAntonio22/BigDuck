
\chapter{Quick Reference}

\section{Enviroment}

\paragraph{} Welcome to the BigDuck programming language reference. Through
this chapter it is going to be presented all the syntax and features present
on this programming language.

\paragraph{} Once downloaded the codebase, on any UNIX-like environment (like
macOS or Linux) you can use Make to build the compiler. Just be sure you have
installed ANTLR 4.9 on its usual directory \texttt{/usr/local/lib/}. However
if you are on macOS Monterey, it is almost certain that you can run the
\texttt{duck} executable like any other executable from the terminal.

\paragraph{} After getting the compiler, create a new text file with the
\texttt{.duck} file extension, and type the following text.
\begin{verbatim}
proc main() {
    print("Hello, World!");
}
\end{verbatim}
\paragraph{} Every BigDuck program starts by the last procedure declared
(procedures will be explained in more detail further in the chapter). The print
command displays on screen the text inside the quotation marks.

\paragraph{} Run this program with the following commands.
\begin{verbatim}
./duck hello.duck
./duck run hello.quack
\end{verbatim}
\paragraph{} The first command compiles the source code and creates a new file,
called executable, with the same name of the source code file just with the
extension changed to \texttt{.quack}. The second command will read the file
and execute it.

\section{Variables}
\paragraph{} To work with values it is necessary to store them in variables.
Variables can be thought of containers for values in memory, therefore, you
can used them to make any desired computation.

\paragraph{} Look at the following example for variable declaration.
\begin{verbatim}
proc main()
    var a, b, c int;
    var x, y float;
    var condition bool;
{
    print(a, b, c);     #| prints: 0 0 0 |#
    print(x, y);        #| prints: 0 0   |#
    print(condition);   #| prints: false |#
}
\end{verbatim}

\paragraph{} As you can see you have to start with the keyword \texttt{var}
followed by a list of names separated by comas, and a type keyword. This tells
to the language that every name on the list will be of the same type.

\paragraph{} On the BigDuck language there are 3 primitive types; \texttt{int},
\texttt{float}, and \texttt{bool}. Which are enough for any kind of numeric
and logic operation.

\paragraph{} The text that is enclosed by \texttt{\#|} and \texttt{|\#} is
ignored by the compiler, this are called comments and are used to clarify a
section of code. In this case they show the output of performing such
instructions.

\paragraph{} On the BigDuck language all variables are initialize to their
respective zero value, for ints and floats is \texttt{0}, and for bools is
\texttt{false}. The next section we will discuss on how to change this values
and work with variables.

\section{Statements}
\subsection{Assignments}
\subsection{Arithmetic Expressions}
\subsection{Operator Hierarchy}

\section{Conditional Statements}

\section{Loop Statements}
\subsection{Infinite Loop}
\subsection{While Loop}
\subsection{For Loop}
\subsection{Control Flow Statements}

\section{Procedures}

\section{Tensorial Types}

\section{Built-in Procedures}
