
\chapter{Quick Reference}

\section{Enviroment Setup}

\paragraph{} Welcome to the BigDuck programming language reference. Through
this chapter it is going to be presented all the syntax and features present
on this programming language.

\paragraph{} Once downloaded the codebase, on any UNIX-like environment (like
macOS or Linux) you can use Make to build the compiler. Just be sure you have
installed ANTLR 4.9 on its usual directory \texttt{/usr/local/lib/}. However
if you are on macOS Monterey, it is almost certain that you can run the
\texttt{duck} executable like any other executable from the terminal.

\paragraph{} After getting the compiler, create a new text file with the
\texttt{.duck} file extension, and type the following text.
\begin{verbatim}
proc main() {
    print("Hello, World!");
}
\end{verbatim}
\paragraph{} Every BigDuck program starts by the last procedure declared
(procedures will be explained in more detail further in the chapter). The print
command displays on screen the text inside the quotation marks.

\paragraph{} Run this program with the following commands.
\begin{verbatim}
duck hello.duck
duck run hello.quack
\end{verbatim}
\paragraph{} The first command compiles the source code and creates a new file,
called executable, with the same name of the source code file just with the
extension changed to \texttt{.quack}. The second command will read the file
and execute it.

\section{Variables}
\paragraph{} To work with values it is necessary to store them in variables.
Variables can be thought of containers for values in memory, therefore, you
can used them to make any desired computation.

\paragraph{} Look at the following example for variable declaration.
\begin{verbatim}
proc main()
    var a, b, c int;
    var x, y float;
    var condition bool;
{
    print(a, b, c);     #| prints: 0 0 0 |#
    print(x, y);        #| prints: 0 0   |#
    print(condition);   #| prints: false |#
}
\end{verbatim}

\paragraph{} As you can see you have to start with the keyword \texttt{var}
followed by a list of names separated by commas, and closed by type keyword.
This tells to the language that every name on the list will be of the same type.

\paragraph{} On the BigDuck language there are 3 primitive types; \texttt{int},
\texttt{float}, and \texttt{bool}. Which are enough for any kind of numeric
and logic operation.

\paragraph{} The text that is enclosed by \texttt{\#|} and \texttt{|\#} is
ignored by the compiler, this are called comments and are used to clarify a
section of code. In this case they show the output of performing such
instructions.

\paragraph{} On the BigDuck language all variables are initialize to their
respective zero value, for ints and floats is \texttt{0}, and for bools is
\texttt{false}. The next section we will discuss on how to change this values
and work with variables.

\section{Statements}

\paragraph{} On any computational language exists the notion of
\emph{sequencing}, this could be for instructions, operations, functions, etc.
This sequencing mechanism allows us to indicate the order and steps to be taken
by an algorithm.

\subsection{Assignments}

\paragraph{} After the declaration of a variable, the assignment operator
\texttt{<-} allows to indicate a value to be hold by the variable. It will
remain this value untill another asignment is performed.

\subsection{Arithmetic Expressions}

\paragraph{} In order to perform operations on values or variables, there are
several operators that can be used for different purposes. For example take a
look at the following program.

\begin{verbatim}
proc main()
    var a, b float;
{
    a <- 1;
    b <- 1;
    print(a + b);   #| prints: 2 |#
    a <- 1 + b;     #| now a holds the value: 2 |#
    b <- 5 * b;     #| now b holds the value: 5 |#
    print(a / b);   #| prints: 2 / 5 = 0.4 |#
}
\end{verbatim}

\subsection{Operator Precedence and Associativity}

\paragraph{} As in mathematics, the order of operations is important for certain
operations, thus it is advice to have into consideration the following table.
The earlier the operator appear on the table, the higher is its precedence. All
Operator are left to right associative to provide a natural left to right
reading.

\begin{figure}[h]
    \centering
    \begin{tabular}{cc}
        \toprule
        \textbf{Operator} & \textbf{Usage} \\
        \midrule \texttt{()} & \texttt{(expression)}\\
        \midrule \texttt{*}, \text{/} & \texttt{a * b}, \texttt{a / b}\\
        \midrule \texttt{+}, \text{-} & \texttt{a + b}, \texttt{a {-} b}\\
        \midrule
        \texttt{=}, \texttt{/=}, \texttt{<},
        \texttt{>}, \texttt{<=}, \texttt{<=}, &
        \texttt{a <relation> b}\\
        \midrule \texttt{not} & \texttt{not a}\\
        \midrule \texttt{and} & \texttt{a and b}\\
        \midrule \texttt{or} & \texttt{a or b}\\
        \midrule \texttt{<-} & \texttt{a <- b}\\
        \bottomrule
    \end{tabular}
\end{figure}

\paragraph{} As you can see multiplication and division, addition and
substraction, or relational operators have the same precedence. The order of
evaluation is resolved by giving priority to one that was read first.

\paragraph{} Therefore the expression \texttt{a {+} b {-} c} is equal to
\texttt{(a {+} b) {-} c}, and it is \textbf{not equal} to
\texttt{a {+} (b {-} c)}.

\section{Conditional Statements}

\paragraph{} On any computational language exists the notion of
\emph{decisions}. The decision mechanism is use to perform certain instructions
under certain conditions.

\paragraph{} The BigDuck language allows for decisions to be taken during
program execution. For an example take a look at the following program.

\begin{verbatim}
proc main()
    var a, b int;
{
    a <- read("Type a value for a");
    b <- read("Type a value for b");

    if a = b {
        print("a equals b");
    } else {
        print("a does not equals b");
    }
}
\end{verbatim}

\paragraph{} The first two instructions are a especial syntax to indicate that
the user can give a value and assign it to a variable. Despite these looking
like the value obtained by read is assigned to the variable, the whole line
is the read and assigment, therefore no operation can be inmediately applied to
a read value. This desicion was taken to enforce legibility.

\paragraph{} Whether the given values for a and b are equal or not, the program
will print a diffent message. The first print is performed when the if clause
condition holds true, otherwise the else clause will be perfomed.\\

\noindent
Else clauses can be omitted like here.

\begin{verbatim}
proc main()
    var a, b int;
{
    a <- read("Type a value for a");
    b <- read("Type a value for b");

    if a = b {
        print("a equals b");
    }
}
\end{verbatim}

\newpage

\noindent
And you can stack if else clauses for multiple cases.

\begin{verbatim}
proc main()
    var a, b int;
{
    a <- read("Type a value for a");
    b <- read("Type a value for b");

    if a < b {
        print("a is less than b");

    } else if a > b {
        print("a is greater than b");

    } else {
        print("a equals b");
    }
}
\end{verbatim}

\section{Loop Statements}

\paragraph{} On any computational language exists the notion of \emph{loops}.
The looping mechanism allow us to concisely tell the computer to perform
some operations $n$ amount of times. Otherwise we would have to sequence $n$
amount of times the same instruction and this would not be managable on the
long run (also some computations explicitly require a looping mechanism).

\subsection{Infinite Loop}
\noindent
The easiest way to loop instructions is shown by the next example.

\begin{verbatim}
proc main() {
    loop {
        print("Hello, World!");
    }
}
\end{verbatim}

\paragraph{} If you run this program you will see that it never ends and it is
going to fill the console with \texttt{``Hello, World!''}, to stop the
program type \texttt{ctrl} + \texttt{c}. The infinite loop is useful for
interactive programs, where it is up to the user when to end the program.

\paragraph{} There are other mechanisms to stop an infinite loops, they will be
covered in depth on Subsection~\ref{ctrl_flow}.

\subsection{While Loop}

\paragraph{} Most of the times it is desired to end a loop when a condition is
met, thus the BigDuck language provide syntax for handling this kind of loops.
A while loop is a type of loop that continues until the loop condition is false.

\begin{verbatim}
proc main()
    var i int;
{
    i <- 1;

    loop i < 10 {
        print("i value is", i);
        i <- i + 1;
    }
}
\end{verbatim}

\paragraph{} The previous program prints all numbers starting from 1 up to 9,
10 will not be printed because the condition \texttt{i < 10} is no longer true.

\subsection{For Loop}

\paragraph{} Many times when looping, there is going to be a control variable,
for example on the previous program it was \texttt{i}. Since this is really
common there is and special syntax for this kind of loop. The following program
is equivalent to the previous one, however is using the for loop syntax.

\begin{verbatim}
proc main()
    var i int;
{
    loop i <- 1; i < 10; i <- i + 1 {
        print("i value is", i);
    }
}
\end{verbatim}

\paragraph{} As you can it is much more compact and explicit on how the loop
will behave. The first statements is an assigment to initialize the control
variable (is only performed once), the second statement is the loop condition,
and the third statement is an assigment statement to modify the control
variable.

\paragraph{} At this point you may think that the while syntax is redundant
against the for syntax. However this is not true, the recommendation is to
use the while syntax when you are not sure how many iterations is going to take
the loop. On the other side, for style syntax is best when you know beforehand
how is the loop going to behave.

\subsection{Do While Loop}

\paragraph{} The next type of loop does not have a dedicated syntax, however
it is also a common type of loop present on many programming languages. 
Therefore here is provided an idiom to achieve the same goal.

\begin{verbatim}
proc main()
    var i int;
    var ok bool;
{
    loop ok <- false; not ok; ok <- i = 2 {
        i <- read("Type 2 to exit the loop");
    }
}
\end{verbatim}

\paragraph{} Do while loops are at least run one time, by assuming that it is
not ok to exit the loop the program will enter inside the loop. And at every
iteration it is tested whether the condition to exit is met.

\subsection{Control Flow Statements}\label{ctrl_flow}

\paragraph{} Throughtout loops sometimes it is required to exit earlier or
go to the next iteration. This could be achieve by using logic, however it can
be messy and obfuscate the meaning of the code. Take a look at the following
program.

\begin{verbatim}
proc main() {
    loop #| Exit condition |# {
        if #| Condition skip iterations|# {
            skip;
        }

        if #| Condition to break from loop |# {
            break;
        }
    }
}
\end{verbatim}

\paragraph{} When a skip statement is reached, the following code on the loop
will be ignored and the next iteration is going to be reached. However the
break statement exits the loop. These are usefull to avoid unnecessary 
operations or to make the code more readable.

\newpage

\section{Procedures}

\paragraph{} At this point everything has been done inside the main procedure,
this means that all code written in main belongs to the same context. With
context it is meant that all variables and flow of the program is self contained
in the procedure. This may be good enough for small programs, however as the
complexity increases you start seeing code repetition or related code to perform
an action.

\paragraph{} Many programming languages, including BigDuck, are able to
provide different contexts through; functions, procedures, methods, etc. On this
language it is done through procedure, whose syntax is the following.

\begin{verbatim}
proc name(args) -> type
    vars
{
}
\end{verbatim}

\paragraph{} Where \texttt{name} is the procedure name, \texttt{args} is a list
of arguments to be passed to a procedure, \texttt{type} indicates the return
type of the procedure, and \texttt{vars} indicate local variables to be use.\\

\noindent
The following are examples of procedures.

\begin{verbatim}
proc square(x float) -> float {
    return x * x;
}

proc distance(x1, y1, x2, y2 float) -> float {
    return sqrt(square(x2 - x1) - square(y2 - y1));
}

proc close_enough(x1, y1, x2, y2 float) -> bool {
    return distance(x1, y1, x2, y2) > 0.01;
}

proc get_circle_area(r float) -> float {
    return 3.14159 * square(r);
}
\end{verbatim}

\paragraph{} Using procedures is not only helpful to write less code, but it
also helps to make the code more abstract (hiding unnecessary information), and
it allows to handle problems with more ease.

\newpage

\paragraph{} The last thing to cover about procedures is recursion. Recursion
can be seen as a higher level form of loops, this is because it covers all the
use cases of loops and can be a cleaner solution. However sometimes may be
harder to think in a recursive solution or viceverse, thus it remains at the
taste of the user to decide when to use recursion or loops.

\paragraph{} A classic use case for recursion is for the implementation of the
factorial function. In mathematics the factorial function is defined the
following way.

\[
    f: \mathbb{N}\to\mathbb{N}
    \Rightarrow
    f(x) =
    \begin{cases}
        1                &\text{if } n \leq 0\\
        n \cdot f(n - 1)
    \end{cases}
\]

\paragraph{} The iterative approach (the one using loops) is not as easy to
understand and make take sometime to realize that it computes the same function.

\begin{verbatim}
proc factorial(n int) -> int
    var prod int;
{
    prod <- 1;
    
    loop n > 0 {
        prod <- prod * n;
    }

    return prod;
}
\end{verbatim}

\noindent
However the recursive definition resembles better the mathematical definition.

\begin{verbatim}
proc factorial(n int) -> int {
    if n <= 0 {
        return 1;
    } else {
        return n * factorial(n - 1);
    }
}
\end{verbatim}

\newpage

\section{Tensorial Types}

\paragraph{} As your programs get more complex, you will notice that certain
variables are related to other variables. Many programming languages offer
many tools for handling complex relationships for data modelling.

\paragraph{} However the BigDuck language provides the simplest kind of
structured data called tensors, commonly refered as arrays or multidimensional
arrays. The name was picked from mathematics since it generalizes the notion of
an ordered tuple that can be indexed (known as vectors).\\

\noindent
Look at the following program, which defines a 1-D tensor.

\begin{verbatim}
proc main()
    var a [5]int;
{
    a[1] <- 12;
    print(a[1])     #| prints: 12 |#
}
\end{verbatim}

\paragraph{} This syntax indicates the language that the variable \texttt{a}
can be accessed by values from 0 and less than 5, therefore indexing by 5
produces an error. If no index was given, it will be given the first value of
the tensor.\\

\noindent
For higher dimensional tensors just keep adding more dimensions like here.

\begin{verbatim}
proc main()
    var a [2][5]int;
{
    a[0][1] <- 12;
    print(a[0][1])     #| prints: 12 |#
}
\end{verbatim}

\newpage

\section{Built-in Procedures}
