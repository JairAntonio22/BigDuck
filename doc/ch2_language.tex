
\chapter{Language}

\section{General Overview}
\subsection{Language Name}
\paragraph{} The for the programming language was given as a small joke, one of
the homeworks on the semester was to develop a scanner and parser for a small
language called LittleDuck. Therefore BigDuck could be considered as the next
step for the previous mentioned language, even though there is no similarities
but the name between these languages.

\paragraph{} Additionaly to this, I really like birds and use them as a naming
scheme for my devices, thus the decision seemed natural and adecuate.

\subsection{Main Features Description}
\paragraph{} BigDuck is language aimed for the developement of mathematical
models commonly used in Machine-Learning and Data Science. Therefore this
language includes integer and floating point arithmetic, vector and matrix
operations, and some basic utilities for reading and writing \texttt{.csv} 
iles. All this to make it easier for the user to work within the
Machine-Learning and Data Science fields.

\newpage

\section{Language Errors}
\subsection{Compile-Time Errors}
\paragraph{} BigDuck is language aimed for the developement of mathematical
models commonly used in Machine-Learning and Data Science. Therefore this
language includes integer and floating point arithmetic, vector and matrix
operations, and some basic utilities for reading and writing \texttt{.csv} 
iles. All this to make it easier for the user to work within the
Machine-Learning and Data Science fields.

\begin{figure}[h]
    \centering
    \begin{tabular}{p{1.5in}p{2.5in}}
        \toprule
        \textbf{Error message} & \textbf{Description}\\

        \midrule Duplicate symbol &
        This occurs when the current symbol is already used on the declared
        scope.\\

        \midrule Variable was not \newline declared &
        This occurs when the current variable has not been \newline previously
        declared.\\

        \midrule Procedure was not \newline declared &
        This occurs when current procedure has not been \newline previously
        declared.\\

        \midrule Void procedure used \newline in expression &
        This occurs when there is no return value on the procedure call in
        an expression.\\

        \midrule Procedure expected $n$ \newline arguments, given $m$ &
        This occurs when the procedure call was given $m$ arguments but it
        does not match with expectected $n$ atributes specified on declaration.\\

        \midrule Type error mismatch &
        This occurs when an operation can not be performed with the given
        operands.\\

        \midrule Expected boolean \newline expression &
        This occurs when an expression inside a condition (if's or loop's) does
        not evaluated to a boolean value.\\

        \midrule Parameter expected \newline to be $a$, given $b$ &
        This occurs when the parameter of type $b$ does not match with type $a$
        expected by the procedure.\\

        \midrule Return type \newline different from \newline procedure sign &
        This occurs when the returned value does not match with the return
        type expected.\\

        \midrule Tensor dimension must be constant &
	This occurs when a tensor is declared with variable dimension.\\

        \midrule Tensor dimension must be greater than 0 &
	This occurs when a tensor is declared with a not valid dimension.\\

        \bottomrule
    \end{tabular}
\end{figure}

\newpage

\begin{figure}[h]
    \centering
    \begin{tabular}{p{1.5in}p{2.5in}}
        \toprule
        \textbf{Error message} & \textbf{Description}\\

        \midrule Index value must be of type int &
	This occurs when the index for a tensor does not resolve into an
	integer value.\\

        \midrule Tensor access does not match with dimensions &
	This occurs when the number of indexes for a tensor does not match
	with the declared dimensions.\\

        \midrule Scalar value cannot be indexed &
	This occurs when it is attempted to index a scalar variable.\\

        \bottomrule
    \end{tabular}
\end{figure}

\subsection{Run-Time Errors}
\begin{figure}[h]
    \centering
    \begin{tabular}{p{1.5in}p{2.5in}}
        \toprule
        \textbf{Error message} & \textbf{Description}\\

        \midrule Local address used \newline in data segment &
        This occurs when it is attempted to initialize a local address before
        having a local context setup.\\

        \midrule Invalid address used \newline in data segment &
        This occurs when it is attempted to initialize an address that does not
        conform to with address specification.\\

        \midrule Unexpected operator \newline at data segment &
        This occurs when an operator is used on a segment it was not supposed
        to be.\\

        \midrule Unexpected operator &
	This occurs when an the virtual machine cannot recognized a given 
	operator.\\

        \midrule Type error mismatch &
        This occurs when an operation can not be performed with the given
        operands.\\

        \bottomrule
    \end{tabular}
\end{figure}

