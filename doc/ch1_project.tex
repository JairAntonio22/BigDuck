
\chapter{Project}

\section{Introduction}
\subsection{Purpose}
\paragraph{} This document describes the software developement process, 
technical documentation, and user manual, for the final project of the Compiler
Design course. Which consists on the design and implementation of a programming
language and a virtual machine.

\subsection{Scope}
\paragraph{} The programming language developed is specified to be a compiled
imperative, with support of modules and structured types. Additionaly it is
required to develop a virtual machine capable to execute the output code
generated by the compiler.

\newpage

\section{Software Requirements}
\subsection{Analysis}
\paragraph{} Based on the specifications and recomendations given by the
teachers, the following requirements were defined as necessary for the
successful development of this project.

\paragraph{Functional requirements}
\begin{enumerate}
    \item The programming language must aim to solve a domain specific problem.
    \item The compiler must support scoped and global variables.
    \item The compiler must support numeric data types.
    \item The compiler must support conditional statements.
    \item The compiler must support loop statements.
    \item The compiler must support modules.
    \item The compiler must support recursion.
    \item The compiler must support structured types.
    \item The compiler must report compile-time errors.
    \item The compiler must generate intermidiate code.
    \item The virtual machine must execute generated code.
    \item The virtual machine must manage program memory.
    \item The virtual machine report run-time errors.
\end{enumerate}

\paragraph{Non-Functional requirements}
\begin{enumerate}
    \item The language grammar must be non-ambiguous.
    \item The compiler shall use a scanner and parser generation tool.
    \item The compiler must be efficient in time and memory.
    \item The virtual machine must be efficient in time and memory.
\end{enumerate}

\newpage

\subsection{Test Cases}

\section{Software Developement Process}

\subsection{Developement Process Description}
\paragraph{} The project was developed throught weekly sprints, were each
sprint consisted in developing a major feature needed for the programming
language compilation or execution. It must be said that despite having an
suggested schedule, the reality is that the project went a little bit different
from this schedule. This is because some features were prioritize to be
implemented first.

\subsection{Weekly Log}

\begin{figure}[h]
    \centering
    \begin{tabular}{ccp{3in}}
        \toprule
        \textbf{Week no.} & \textbf{Date} & \textbf{Description}\\
        \midrule
        0 & Sep 20 & Proposal Developement\\
        1 & Sep 27 & Lexic and syntax analysis\\
        2 & Oct  4 & Symbol table and sematic cube\\
        3 & Oct 11 & Expressions compilation\\
        4 & Oct 18 & Conditionals compilation\\
        5 & Oct 25 & Loops compilation\\
        6 & Nov  1 & Procedures compilation\\
        7 & Nov  8 & Sematic analysis, memory map,
                     \newline and virtual machine\\
        8 & Nov 15 & Structured types compilation,
                     \newline and application specific code\\
        \bottomrule
    \end{tabular}
\end{figure}

\subsection{Git Commitments}

